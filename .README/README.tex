\documentclass{article}

%   included packages
\usepackage[a4paper,margin = 14mm]{geometry} %0.5 inch margins a4 paper
\usepackage{amsmath}
\usepackage{newpxtext,newpxmath}
\usepackage{microtype}
\usepackage{titling}

%   version history (should be used in conjunction with git)
\usepackage{vhistory}

%   Allows me to use hyperlinks for table of contents etc.
\usepackage{hyperref}
\usepackage{xcolor}
\hypersetup{
    linktoc     =   all,
    colorlinks,
    citecolor   =   black,
    filecolor   =   black,
    linkcolor   =   black,
    urlcolor    =   blue
}

%   Setting up the code examples
% \usepackage{listings}


% \definecolor  {dkgreen}   {rgb}   {0,0.6,0}
% \definecolor  {gray}      {rgb}   {0.5,0.5,0.5}
% \definecolor  {mauve}     {rgb}   {0.58,0,0.82}

% \lstset{
%   frame               =   tb,
%   language            =   C++,
%   aboveskip           =   3mm,
%   belowskip           =   3mm,
%   showstringspaces    =   false,
%   columns             =   flexible,
%   basicstyle          =   {\small\ttfamily},
%   numbers             =   none,
%   numberstyle         =   \tiny\color{gray},
%   keywordstyle        =   \color{blue},
%   commentstyle        =   \color{dkgreen},
%   stringstyle         =   \color{mauve},
%   breaklines          =   true,
%   breakatwhitespace   =   true,
%   tabsize             =   3
% }

%   formatting for document

%   Push everything to the left cause I think it looks cleaner
\preauthor  {\begin{flushleft}}
\postauthor {\end{flushleft}}
\predate    {\begin{flushleft}}
\postdate   {\end{flushleft}}    
\setcounter {secnumdepth}{0}

%   Document Title & author
\title      {\Huge Sasso\_C\_Cpp\_GenNetLib}
\author     {By: Anthony Sasso}

%   updates date for each compile, that way this + versioning tells reader which they are reading?
\date       {\today}

%   no indents for paragraphs (change depending on doc type)
\setlength  {\parindent}    {0pt}

%   custom commands / formatting shortcuts
%   combine bold & italic
\newcommand{\textbfit}[1]{\textbf{\textit{#1}}} 

%   footnote with label, colour, and inputted text... then sets back to default
% \newcommand{\clfootnote}[3]{\color{#2}{\footnote{\label{#1}{ \color{#2}{#3}}}}\color{defaultcolor}} 

%   set to defualt colour (black)
\AtBeginDocument    {\colorlet{defaultcolor}{.}}

%   writing goes here
\begin{document}

\maketitle

\tableofcontents

\newpage
\section{Introduction}
Collection of generic networking libraries in C/C++ ideally simplifying use and allowing quicker project builds.

\section{Getting Started}
\subsection{Software Dependencies}
\begin{enumerate}
    \item Ensure C17, Cxx20, CMake, Ninja, Clang, Doxygen are present on target machine for development and compilation.
    \item Docker desktop / docker instance exists on your target machine for using docker compose \& docker functions.
    \item Have an active internet connection for pulling the server base images, etc during compose.
    \item Optionally have VSCode or some automation tool for using CMake > Ninja-Multi-Config generation (please ensure to use the Multi-Config version for testing etc. \textbf{before} pushing back to your branch).
\end{enumerate}

\section{Build \& Test}
\begin{enumerate}
    \item Download / Clone Target Branch (by default would recommend the latest one).
    \item Have Docker up / running.
    \item Run the appropriate CMake build for the target (Ninja-Multi-Config).
    \item Run \textbf{``docker-compose up -d''} to run the test server.
    \item Access at test output file.
\end{enumerate}

\section{Contribute!}
To Contribute to the project follow these rules:
\begin{itemize}
    \item First create an issue / feature request notifying any changes that are wanted.
    \item Wait to hear the initial response / accepting of that issue before you begin coding.
    \item Fork or Branch (dependent on scope, membership in core development team, etc.) the repo and code any necessary fixed to fulfill this new feature / fix the bug.
    \item Issue a pull request back to the main repo with your latest released version and attach a text file or description within the pull request detailing your changes, additions, considerations with this pull request. In addition, follow the ``LatexToMarkdownNoted.md'' within the ``.README'' Source folder to update the README listing any necessary changes, and have this as your final commit before the pull request.
\end{itemize}

\end{document}
